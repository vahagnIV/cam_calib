\documentclass[a4paper,10pt]{article}
\usepackage[utf8]{inputenc}
\usepackage{amsmath}
\usepackage{mathtools}
\usepackage{setspace}
\usepackage{indentfirst}
%opening
\title{}
\author{}

\begin{document}

\maketitle

\begin{abstract}

\end{abstract}

\section{Camera geometry}

Camera Matrix has the following form and depends only on the device itself.

\begin{equation}
 K=\left(\begin{array}{ccc}
          f_x & s & c_x \\
          0 & f_y & c_y \\
          0 & 0 & 1
         \end{array}
\right)
\end{equation}
Pattern coordinates are defined as ${\bf C}^k$.

We perform $M$ measurements enumerated by index $\mu = 0,\ldots, M-1$. For $\mu^{th}$ experiment we have the transformation matrix ${\bf R}_\mu$ and a translation vector ${\bf T}_\mu$ that transform the pattern points ${\bf C}$ to the camera coordinate system.

\begin{equation}
 {\bf X}^k_\mu = {\bf R}_\mu {\bf C}^k+ {\bf T}_\mu
\end{equation}

The corresponding points are obtained via camera projection matrix as follows
{\setstretch{1.7}
\begin{equation}
 {\bf U}^k_\mu = {\bf K} {\bf X}^{k}_\mu \equiv {\bf K} \left({\bf R}_\mu {\bf C}^k + {\bf T}_\mu\right) \equiv \left(\begin{array}{c}
  u^{k}_\mu \\ v^{k}_\mu \\ w^{k}_\mu
 \end{array}\right)\label{uproj}
\end{equation}
}
\begin{equation}
 x^{k}_\mu = u^{k}_\mu / w^{k}_\mu, \quad y^{k}_\mu = v^{k}_\mu / w^{k}_\mu
\end{equation}

Let us define $\tilde{x}, \tilde{y}$ - distorted points and

\begin{equation}
 \rho^{(k)} = \sqrt{x^2 + y^2}
\end{equation}

Radial distortion:
{\setstretch{1.7}
\begin{equation}
\begin{array}{c}
 \tilde{x} = x + x(k_1 \rho^2 + k_2 \rho^4 + k_3 \rho^6) \\
 \tilde{y} = y + y(k_1 \rho^2 + k_2 \rho^4 + k_3 \rho^6)
\end{array}
\end{equation}
}

Skew distortion:
{\setstretch{1.7}
\begin{equation}
 \begin{array}{c}
  \tilde{x} = x + 2 p_1 xy + p_2 \left(\rho^2 + x^2\right) \\
  \tilde{x} = y + 2 p_2 xy + p_1 \left(\rho^2 + y^2\right)
 \end{array}
\end{equation}
}

So, we have
{\setstretch{1.7}
\begin{equation}
\begin{array}{c}
 \tilde{x}^k_\mu\equiv \tilde{x}^k_\mu\left(R_\mu, T_\mu, C^k, K, k_1,k_2,k_3, p_1,p_2\right) \\
 \tilde{y}^k_\mu\equiv \tilde{y}^k_\mu\left(R_\mu, T_\mu, C^k, K, k_1,k_2,k_3, p_1,p_2\right) \\
\end{array} 
\end{equation}
}
The error functions ${\bf e}^k_\mu$ read:

{\setstretch{1.7}
\begin{equation}
 {\bf e}^k_\mu = \left(
 \begin{array}{c}
                  \operatorname{measured}(x^k_\mu) - \tilde{x}^k_\mu \\
                  \operatorname{measured}({y}^k_\mu) - \tilde{y}^k_\mu \\
                 \end{array}\right)
\end{equation}
}

And the function we have to minimize reads:

\begin{equation}
 L = \sum\limits_{k,\mu} {\bf e}^k_\mu {\bf e}^k_\mu
\end{equation}

\section{Calibration on a flat chessboard pattern}

Let us consider a special case when the points ${\bf C}^k$ are the corners of a chessboard pattern:

\begin{equation}
 {\bf C}^{pq}\equiv {\bf C}^k= a\left(\begin{array}{c}
                  p \\
                  q \\
                  0
                 \end{array}
\right)
\end{equation}
where $p=0,\ldots, N_x-1$ and $q=0,\ldots,N_y-1$ enumerate corners in the pattern, $k = qN_x + p$ is unified index and $a$ is the length of the pattern's square size in any units of measure.

Since, on the pattern plane in its own coordinate system $Z=0$, the last column of ${\bf R}$ does not play any role, and, the transformation \eqref{uproj} can be rewritten as:

\begin{equation}
 {\bf U}^k_\mu = {\bf H}_\mu {\tilde{\bf  C}}^k, \quad {\tilde{\bf  C}}^k = \left(\begin{array}{c}
c^k_1 \\                                                                               
c^k_2 \\
1
\end{array}
\right)
\end{equation}
where
\begin{equation}
 {\bf H} = \lambda {\bf K}\cdot\left(
 \begin{array}{ccc}
  r_{11} & r_{12} & t_1 \\
  r_{21} & r_{22} & t_2 \\
  r_{31} & r_{32} & t_3 \\
 \end{array}
 \right)\equiv\left(
 \begin{array}{ccc}
  h_{1} & h_{2} & h_3 \\
  h_{4} & h_{5} & h_6 \\
  h_{7} & h_{8} & h_9 \\
 \end{array}
 \right) 
 %\equiv [{\bf h}_1 \quad {\bf h}_2 \quad {\bf h}_3] 
 \label{homography}
\end{equation}
is a homography matrix that maps the pattern plane onto the image plane. Here $\lambda$ is an arbitrary constant. Here we dropped the index $\mu$ for readability.

Each point correspondence $U^k_\mu\to C^k_\mu$ gives us a constraint on the elements of ${\bf H}$



Multiplying \eqref{homography} from the left by ${\bf K}^-1$ and imposing that ${\bf r}_1$ and ${\bf r}_2$ are orthonormal gives us 2 constraints on the homography matrix:

{\setstretch{1.7}
\begin{equation}
\begin{array}{c}
 {\bf h}_1^T {\bf K}^{-T} {\bf K}^{-1}{\bf h}_2 = 0 \\
 {\bf h}_1^T {\bf K}^{-T} {\bf K}^{-1}{\bf h}_1 = {\bf h}_2^T {\bf K}^{-T} {\bf K}^{-1}{\bf h}_2
\end{array} \label{bconstraints}
\end{equation}
}

Let us define 

{\setstretch{1.7}
\begin{equation}
 {\bf B} = {\bf K}^{-T} {\bf K}^{-1} = \left(
 \begin{array}{ccc}
 \frac{1}{f_x^2} & -\frac{s}{f_x^2 f_y} & \frac{c_y s - c_x f_y}{f_x^2f_y}\\
 -\frac{s}{f_x^2 f_y} & \frac{s^2}{f_x^2f_y^2} + \frac{1}{f_y^2} & -\frac{s(c_y s - c_x f_y)}{f_x^2 f_y^2} - \frac{c_y}{f_y^2} \\
 \frac{c_y s - c_x f_y}{f_x^2 f_y} & - \frac{s(c_y s - c_x f_y)}{f_x^2 f_y^2} - \frac{c_y}{f_y^2} & \frac{(c_y s - c_x f_y)^2}{f_x^2 f_y^2} + \frac{c_y^2}{f_y^2} + 1
 \end{array}
\right)
\end{equation}
}

\begin{equation}
 {\bf b} = \left(B_{11},B_{12}, B_{22}, B_13, B_{23}, B_{33}\right)^T
\end{equation}

With this notations the constraints \eqref{bconstraints} can be rewritten as follows:

% \begin{equation}
% {\bf V} = \left(
%  \begin{array}{cccccc}
%   h_{11}h_{12} & h_{12}h_{21} + h_{11}h_{22}  & h_{21}h_{22} & h_{12}h_{31} + h_{11}h_{32} & h_{31}h_{22} + h_{21}h_{32} & h{31}h_{32} \\
%   h_{11}^2 - h_{12}^2 & 2(h_{11}h_{21} - h_{12}h_{22}) & h_{21}^2 - h_{22}^2 & 2(h_{31}h_{11} - h_{12}h_{32}) & 2(h_{21}h_{31} - h_{22}h_{32}) & h_{31}^2 - h_{32}^2\end{array}\right)
% \end{equation}

\begin{equation}
 {\bf V}{\bf b} = 0,
\end{equation}
where
{\setstretch{1.7}
\begin{equation}
{\bf V}^T = \left(
\begin{array}{cc}
 h_{11}h_{12} &  h_{11}^2 - h_{12}^2\\
 h_{12}h_{21} + h_{11}h_{22} & 2(h_{11}h_{21} - h_{12}h_{22}) \\
 h_{21}h_{22} & h_{21}^2 - h_{22}^2 \\
 h_{12}h_{31} + h_{11}h_{32} & 2(h_{31}h_{11} - h_{12}h_{32}) \\
 h_{31}h_{22} + h_{21}h_{32} & 2(h_{21}h_{31} - h_{22}h_{32}) \\
 h_{31}h_{32} & h_{31}^2 - h_{32}^2
\end{array}\right)
\end{equation}
}

Stacking $n\geq 3$ equations we can obtain the vector ${\bf b}$ up to an arbitrary scalar.


\section{Homography estimation by 4 point correspondences}

Let us have 4 points $U^k=(u_1, u_2, 1)$ and $X^k = (X_1^k, X_2^k, 1)$ that are related via a homography matrix ${\bf H}$ as follows:

\begin{equation}
 u_\alpha^k = \frac{\left({\bf H}{\bf X}^k\right)_\alpha}{\left({\bf H}{\bf X}^k\right)_3}
\end{equation}

Taking into account the notation \eqref{homography}, this equation can be rewritten as follows:

\begin{equation}
 \left(\begin{array}{ccccccccc}
        X^k_1 & X^k_2 & 1 & 0 & 0 & 0 & -u^k_1 X^k_1 & - u^k_1 X^k_2 & -u^k_1\\
        0 & 0 & 0 & X^k_1 & X^k_2 & 1 & -u^k_2 X^k_1 & -u^k_2 X^k_2 & -u^k_2
       \end{array}
\right)\cdot\left(\begin{array}{c}
                              h_1 \\h_2\\h_3\\h_4\\h_5\\h_6\\h_7\\h_8\\h_9
                             \end{array}
\right) = 0
\end{equation}

Stacking 4 such equations together we get an $8\times 9$ matrix ${\bf L}$ and the combined equations can be rewritten as follows:

\begin{equation}
 {\bf L}{\bf h} = 0
\end{equation}


\end{document}
